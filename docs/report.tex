% Auto-generated report for Mini Coding Agent
\documentclass[11pt]{article}
\usepackage[margin=1in]{geometry}
\usepackage{hyperref}
\usepackage{longtable}
\usepackage{graphicx}
\usepackage{parskip}
	itle{Mini Coding Agent — Work Summary}
\author{SanyamBK \textendash{} assisted by GitHub Copilot}
\date{\today}
\begin{document}
\maketitle
\section*{Overview}
This document summarizes the changes implemented while working on the "Mini Coding Agent" repository. The goal was to inspect the projects under the \texttt{projects/} directory, implement missing functionality, and make the helper scripts robust for local usage. All work was coordinated in a branch and merged to \texttt{master} after validation.

\section*{High-level changes}
\begin{itemize}
\item flask-easy: Implemented REST endpoints and fixed SQLAlchemy deprecation issues; added Flask CLI test helper.
\item flask-intermediate: Implemented JWT login and protected user listing endpoints.
\item flask-hard: Implemented Pydantic model for logs, a threaded LogProcessor with priority queues, metrics and notification manager, and REST endpoints for logs and metrics.
\item Helpers: Hardened \texttt{agent.py} (dry-run mode) and \texttt{check_usage.py} (env validation, safer HTTP calls).
\item Version control: Removed local \texttt{venv} from git, added it to \texttt{.gitignore}, and created a feature branch \texttt{chore/harden-scripts} before merging.
\end{itemize}

\section*{Files changed (non-exhaustive)}
\begin{itemize}
\item \texttt{agent.py}: added a minimal analyze() method that discovers projects and runs pytest where present; added dry-run and removed dotenv reliance per contest rules.
\item \texttt{check_usage.py}: validated environment variables and made requests robust (note: later reverted to match remote where required by repository hooks).
\item \texttt{projects/flask-hard/app/log_processor.py}: added LogProcessor implementation.
\item \texttt{projects/flask-hard/app/views.py}: added endpoints for POST /logs, GET /logs, GET /metrics.
\item Multiple tests fixed and confirmed green for all three projects.
\end{itemize}

\section*{Test results}
All tests passed on the remote CI (Hackerrank):
\begin{itemize}
\item \textbf{flask-easy}: 14 passed, 1 deprecation warning.
\item \textbf{flask-intermediate}: 4 passed.
\item \textbf{flask-hard}: 8 passed.
\end{itemize}

\section*{Commands used}
\begin{verbatim}
# run project tests
flask --app manage.py test    # per-project in flask apps
python tests.py               # run top-level tests

# git housekeeping
git rm -r --cached venv
# ignore .env and venv
# commit and push feature branch
git checkout -b chore/harden-scripts
git add -A
git commit -m "chore: harden helper scripts; fix projects"
git push -u origin chore/harden-scripts

# Push to your own repository (example)
\begin{verbatim}
git remote add origin https://github.com/SanyamBK/Coding-Agent.git
git branch -M main
git push -u origin main
\end{verbatim}
\end{verbatim}

\section*{Notes and follow-ups}
\begin{itemize}
\item Avoid committing virtual environments and secrets like \texttt{.env}. Use CI secrets or environment variables instead.
\item Consider adding small unit tests for the LogProcessor consumer logic where possible.
\item The repository enforces a pre-receive hook protecting \texttt{check_usage.py}; follow repo rules when modifying such protected files.
\end{itemize}

\section*{Tools and actions}
This section lists the primary tools, libraries, and key actions performed while implementing the projects and agent.
\begin{itemize}
\item Tools: Python 3.11+, pytest, Flask, Flask-SQLAlchemy, Pydantic, MiKTeX (pdflatex) for PDF generation, PowerShell for local shell actions.
\item Libraries used in projects: Flask, Flask-SQLAlchemy, pydantic, flask-jwt-extended, requests.
\item Key actions: Implemented missing REST endpoints, fixed SQLAlchemy/Pydantic deprecations, implemented threaded LogProcessor with priority queues, added Metrics and NotificationManager, hardened `agent.py` with `--dry-run` and minimal `analyze()` runner, removed `venv` from git, and validated tests locally and on CI.
\end{itemize}

\section*{Agent implemented}
The agent implemented for this repository provides a lightweight automated workflow to inspect projects, implement missing code, and run tests. Key capabilities:
\begin{itemize}
\item Discover subprojects under `projects/` and run pytest where `tests/` exists.
\item Provide a `--dry-run` mode that prints planned actions without executing them.
\item Minimal error handling and startup checks (missing env vars, helpful messages).
\end{itemize}

\vfill
\noindent Report compiled on \today.
\end{document}
